\documentclass[twoside]{tmlr}

\usepackage{amsmath}
\usepackage{booktabs}
\usepackage{colortbl}
\usepackage{hyperref}
\usepackage{pgfplots}
\usepackage{pgfplotstable}
\usepackage{xcolor}

\pgfplotsset{compat=1.18}
\hypersetup{
  colorlinks=true,
  linkcolor=black,
  citecolor=black,
  urlcolor=black,
  pdfauthor={},
  pdftitle={FX Forecasting Benchmark: Protocol and Governance Summary}
}

\title{FX Forecasting Benchmark: Protocol and Governance Summary}
\author{Anonymous Authors\\Anonymized Institution}
\date{}

\pgfplotstableread[col sep=comma]{data/compute_governance_data.csv}{\governancedata}
\pgfplotstableread[col sep=comma]{data/pcc_reliability_curve.csv}{\pccreliability}
\pgfplotstableread[col sep=comma]{data/pcc_ablation_table.csv}{\pccablation}
\pgfplotstableread[col sep=comma]{data/pcc_regime_gains.csv}{\pccregime}

\begin{document}
\maketitle

\begin{abstract}
We document the protocol, fairness analysis, governance safeguards, and calibration advances for
an anonymised foreign exchange forecasting benchmark. The accompanying supplementary package
includes configuration files, automated evaluation scripts, and lightweight datasets enabling full
reproduction without revealing author identities.
\end{abstract}

\section{Protocol Overview}
The benchmark integrates daily macroeconomic indicators with intraday order book snapshots using a
mixed-frequency data pipeline. Preprocessing enforces deterministic calendaring, leakage-free rolling
splits, and normalisation consistent with the production deployment template. Model configurations for
the mixture-of-experts forecaster and its calibration stack are enumerated in the supplementary
materials, together with a reproducibility script that trains, evaluates, and packages publication
artifacts in a single invocation. Evaluation harnesses hold-out windows with embargoes to align with
industry governance expectations.

\section{Fairness Measures}
Fairness is assessed across the dominant currency blocs using accuracy, true positive rate (TPR), and
false positive rate (FPR). The automated evaluation pipeline exports the summary reported in
Table~\ref{tab:fairness}, demonstrating that the robust configuration narrows cross-region disparities
while improving overall accuracy. Reviewers can regenerate the metrics from the anonymised artifact to
confirm that the fairness conclusions do not rely on institution-specific tooling.

\begin{table}[t]
  \centering
  \caption{Fairness metrics across currency groups using the automated evaluation pipeline.}
  \label{tab:fairness}
  \pgfplotstabletypeset[
    col sep=comma,
    string type,
    every head row/.style={before row=\toprule,after row=\midrule},
    every last row/.style={after row=\bottomrule},
    columns/Scenario/.style={string type,column type=l},
    columns/Group/.style={string type,column type=l},
    columns/Accuracy/.style={column name=Accuracy,column type=r},
    columns/TPR/.style={column name=TPR,column type=r},
    columns/FPR/.style={column name=FPR,column type=r}
  ]{data/fairness_metrics.csv}
\end{table}


\section{Purged Conformal Calibration}
Purged Conformal Calibration (PCC) augments the temporal transformer forecaster with a post-hoc
interval calibration layer that respects market microstructure constraints. Embargoed backtests ensure
that conformal scores are drawn from non-overlapping windows, while recency weighting emphasises the
most recent residuals before quantile extraction. This combination restores the exchangeability
assumptions behind conformal prediction and adapts interval width to evolving volatility regimes.

\subsection{Calibration Quality}
Figure~\ref{fig:pcc-reliability} compares nominal and empirical coverage across confidence levels. The
uncalibrated baseline under-covers by 4--6 percentage points, whereas PCC limits deviation to roughly
one percentage point after applying purging and weighting.

\begin{figure}[t]
  \centering
  \begin{tikzpicture}
    \begin{axis}[
        width=0.85\linewidth,
        height=6cm,
        xmin=0.5,
        xmax=0.9,
        ymin=0.44,
        ymax=0.9,
        xlabel={Nominal coverage},
        ylabel={Empirical coverage},
        legend style={at={(0.02,0.98)},anchor=north west},
        grid=both,
        minor grid style={gray!20},
        major grid style={gray!40}
      ]
      \addplot+[mark=o,thick,color=black] table[x=nominal,y=baseline]{\pccreliability};
      \addlegendentry{Baseline}
      \addplot+[mark=square*,thick,color=gray] table[x=nominal,y=pcc]{\pccreliability};
      \addlegendentry{PCC}
      \addplot[dashed,color=black!60] coordinates {(0.5,0.5) (0.9,0.9)};
    \end{axis}
  \end{tikzpicture}
  \caption{Reliability curve comparing nominal confidence levels with empirical coverage before and
after applying PCC.}
  \label{fig:pcc-reliability}
\end{figure}

\subsection{Ablation Study}
Table~\ref{tab:pcc-ablation} summarises the sequential ablations. Activating embargo logic alone yields
notable reductions in both continuous ranked probability score (CRPS) and coverage error. Adding
exponential decay weights delivers the largest gains, confirming that recency awareness is critical for
handling rapid volatility shifts.

\begin{table}[t]
  \centering
  \caption{Sequential ablations for purged conformal calibration (PCC). Improvements are relative to the uncalibrated baseline.}
  \label{tab:pcc-ablation}
  \pgfplotstabletypeset[
    col sep=comma,
    string type,
    columns={variant,crps,coverage_error,crps_improvement_pct,coverage_improvement_pct},
    columns/variant/.style={column name=Variant,string type,column type=l},
    columns/crps/.style={column name=CRPS,column type=r},
    columns/coverage_error/.style={column name=Coverage Error,column type=r},
    columns/crps_improvement_pct/.style={column name=CRPS $\Delta$ (\%),column type=r},
    columns/coverage_improvement_pct/.style={column name=Coverage $\Delta$ (\%),column type=r},
    every head row/.style={before row=\toprule,after row=\midrule},
    every last row/.style={after row=\bottomrule}
  ]{data/pcc_ablation_table.csv}
\end{table}


The associated improvement breakdown is illustrated in Figure~\ref{fig:pcc-ablation}, where each bar
shows the relative change versus the uncalibrated baseline.

\begin{figure}[t]
  \centering
  \begin{tikzpicture}
    \begin{axis}[
        ybar,
        bar width=18pt,
        width=0.8\linewidth,
        height=6cm,
        ymin=0,
        ymax=12,
        ylabel={Improvement (\%)},
        symbolic x coords={Baseline,PCC + Embargo,PCC + Embargo + Weights},
        xtick=data,
        xticklabel style={rotate=15,anchor=east},
        legend style={at={(0.5,1.05)},anchor=south,legend columns=-1},
        legend cell align=left,
        grid=both,
        minor y tick num=1,
        major grid style={gray!40}
      ]
      \addplot+[fill=black!65] table[x=variant,y=crps_improvement_pct]{\pccablation};
      \addlegendentry{CRPS $\Delta$}
      \addplot+[fill=black!25] table[x=variant,y=coverage_improvement_pct]{\pccablation};
      \addlegendentry{Coverage $\Delta$}
    \end{axis}
  \end{tikzpicture}
  \caption{Relative CRPS and coverage improvements contributed by each PCC component.}
  \label{fig:pcc-ablation}
\end{figure}

\subsection{Regime Robustness}
Figure~\ref{fig:pcc-regimes} reports regime-conditional CRPS gains across calm, volatile, and shock
windows for horizons ranging from fifteen minutes to four hours. The purged calibration maintains
benefits even during stress periods, indicating resilience to abrupt spread widening.

\begin{figure}[t]
  \centering
  \begin{tikzpicture}
    \begin{axis}[
        width=0.85\linewidth,
        height=6cm,
        enlarge x limits=0.2,
        enlarge y limits=0.2,
        symbolic x coords={15m,1h,4h},
        symbolic y coords={Calm,Volatile,Shock},
        y dir=reverse,
        xlabel={Forecast horizon},
        ylabel={Market regime},
        colorbar,
        colormap/viridis,
        point meta min=1.5,
        point meta max=3.5,
        nodes near coords={\pgfmathprintnumber[fixed,precision=2]{\pgfplotspointmeta}\,\%},
        nodes near coords style={font=\small,color=black},
        scatter/use mapped color={draw opacity=1,fill opacity=0.9}
      ]
      \addplot[scatter,only marks,mark=square*,mark size=18pt,scatter src=explicit]
        table[x=horizon,y=regime,meta=crps_gain_pct]{\pccregime};
    \end{axis}
  \end{tikzpicture}
  \caption{Regime-aware CRPS gains (percentage) delivered by PCC. Higher values indicate larger
reductions relative to the baseline forecaster.}
  \label{fig:pcc-regimes}
\end{figure}

\section{Compute Governance}
Figure~\ref{fig:compute-governance} visualises the resource allocation budget for model development,
backtesting, and post-deployment monitoring. Energy and GPU-hour traces originate from governance logs
captured by the automated auditing workflow bundled with the anonymised artifact. Policy constraints
enforce carbon-aware scheduling and cost ceilings, and weekly audits verify adherence to sustainability
thresholds.

\begin{figure}[t]
  \centering
  \begin{tikzpicture}
    \begin{axis}[
        ybar,
        bar width=18pt,
        width=0.8\linewidth,
        height=6cm,
        ymin=0,
        enlarge x limits=0.2,
        symbolic x coords={Training,Evaluation,Monitoring},
        xtick=data,
        ylabel={Resource allocation},
        ymajorgrids=true,
        grid style=dashed,
        legend style={at={(0.5,-0.2)},anchor=north,legend columns=-1},
        legend cell align=left,
        axis y line*=left,
        axis x line*=bottom
      ]
      \addplot+[fill=black!65] table[x=Stage,y=GPUHours]{\governancedata};
      \addplot+[fill=black!25] table[x=Stage,y=EnergyKWh]{\governancedata};
      \legend{GPU hours,Energy (kWh)}
    \end{axis}
  \end{tikzpicture}
  \caption{Compute governance budget allocation across life-cycle stages derived from auditable logs.}
  \label{fig:compute-governance}
\end{figure}

\section{Reproducibility and Governance}
Reproducibility is supported through deterministic data splits, containerised execution, and pinned
dependencies in the supplementary environment specification. The artifact bundles a submission
checklist capturing documentation, fairness audits, and governance validations that reviewers can
inspect without deanonymising the authors. Automated pipelines regenerate tables, figures, and a
machine-readable manifest to facilitate independent verification.

\appendix
\section{Submission Checklist}
The anonymised submission checklist summarises the documentation, fairness analysis, and compute
controls required prior to release. It is distributed alongside the supplementary materials for review.

\end{document}
